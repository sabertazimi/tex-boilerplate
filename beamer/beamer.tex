\documentclass[UTF8,10pt,aspectratio=43]{ctexbeamer} 

% 设置为 Beamer 文档类型,设置字体为 10pt,长宽比为16:9,数学字体为 serif 风格

\batchmode

% 导入一些用到的宏包
\usepackage{amsmath,bm,amsfonts,amssymb,enumerate,epsfig,bbm,calc,color,ifthen,capt-of,multimedia,hyperref}
\usepackage[super,square,comma,sort]{natbib} % for \citet and \citep
\renewcommand{\refname}{参考文献}

\usetheme{Berlin} %主题
\usecolortheme{sustech} %主题颜色

\usepackage{algorithm}  
\usepackage{algorithmicx}  
\usepackage{algpseudocode}
\floatname{algorithm}{算法}
\renewcommand{\algorithmicrequire}{\textbf{输入:}} 
\renewcommand{\algorithmicensure}{\textbf{输出:}}  

\algrenewcommand{\algorithmiccomment}[1]{ $//$ #1}

\usepackage{fancybox}
\usepackage{xcolor}
\usepackage{times}
\usepackage{listings}

\definecolor{mygreen}{rgb}{0,0.6,0}
\definecolor{mygray}{rgb}{0.5,0.5,0.5}
\definecolor{mymauve}{rgb}{0.58,0,0.82}
\lstset{
	backgroundcolor=\color{white},   % choose the background color
	basicstyle=\footnotesize\ttfamily,     % size of fonts used for the code
	columns=fullflexible,
	breaklines=true,                 % automatic line breaking only at whitespace
	captionpos=b,                    % sets the caption-position to bottom
	tabsize=4,
	commentstyle=\color{mygreen},    % comment style
	escapeinside={\%*}{*)},          % if you want to add LaTeX within your code
	keywordstyle=\color{blue},       % keyword style
	stringstyle=\color{mymauve}\ttfamily,     % string literal style
	numbers=left, 
  %	frame=single,
	rulesepcolor=\color{red!20!green!20!blue!20},
	% identifierstyle=\color{red},
	language=c
}

% \usepacakge[UTF8,no-math]{ctex}
% \setCJKsansfont{SimHei} %字体采用黑体  Microsoft YaHei
% \setsansfont{Microsoft YaHei}
% \setmainfont{Microsoft YaHei}

% 题目,作者,学校,日期
\title{XXXXXXX}
\subtitle{\fontsize{9pt}{14pt}\textbf{xxxxxxxxx}}
\author{张伟}
\institute{华中科技大学}
\date{\today}

% 学校Logo
% \pgfdeclareimage[height=0.5cm]{sustech-logo}{sustech-logo.pdf}
% \logo{\pgfuseimage{sustech-logo}\hspace*{0.3cm}}

\AtBeginSection[]
{
	\begin{frame}<beamer>
	\frametitle{\textbf{目录}}
	\tableofcontents[currentsection]
\end{frame}
}
\beamerdefaultoverlayspecification{<+->}
% -----------------------------------------------------------------------------
\begin{document}
% -----------------------------------------------------------------------------

\frame{\titlepage}

\section[目录]{}   %目录
\begin{frame}{目录}
\tableofcontents
\end{frame}

\section{Web 安全漏洞现状综述}
\begin{frame}{Web 安全漏洞现状综述}
\begin{columns}[T] % align columns
  \begin{column}{.40\textwidth}
	  \begin{figure}[thpb]
		  \centering
		  \resizebox{1\linewidth}{!}{
			  % \includegraphics{figures/sustech.pdf}
		  }
		  %\includegraphics[scale=1.0]{figurefile}
		  \caption{SUSTech Campus~\cite{DBLP:conf/osdi/AbadiBCCDDDGIIK16}}
		  \label{fig:campus}
	  \end{figure}
  \end{column}
  
  \hfill

  \begin{column}{.65\textwidth}
    \begin{table}[htbp]
      \caption{Web 安全漏洞}
      \label{tab:rank}
      \centering
			\footnotesize
      \begin{tabular}[c]{l|l}
        \hline
        \multicolumn{1}{c|}{\textbf{占比}} & 
        \multicolumn{1}{c}{\textbf{漏洞}} \\
        \hline
	      37\% & 跨域脚本(Cross-site scripting) \\
	      16\% & SQL 注入(SQL injection) \\
	      5\% & 路径泄露(Path disclosure) \\
	      5\% & 拒绝服务攻击(Denial-of-service attack) \\
	      4\% & 任意代码执行(Arbitrary code execution) \\
        4\% & 内存崩溃(Memory corruption) \\
	      4\% & 跨域请求伪造(Cross-site request forgery) \\
	      3\% & 数据泄露(Data breach/Information disclosure) \\
	      3\% & 任意文件包含(Arbitrary file inclusion) \\
	      2\% & 本地文件包含(Local file inclusion) \\
	      1\% & 远程文件包含(Remote file inclusion) \\
	      1\% & 缓冲区溢出(Buffer overflow) \\
        15\% & 其他 \\
        \hline
      \end{tabular}
    \end{table}
  \end{column}
\end{columns}
\end{frame}

\begin{frame}{State of the art}
  \begin{itemize}
    \item Current anti-procrastination systems lack raw force
    \begin{itemize}
      \item Pomodoro, GTD, etc.
    \end{itemize}
    \item Raw force are often misused and in the wrong hands
    \item People usually try to avoid punishment by telling lies
  \end{itemize}
\end{frame}

\begin{frame}{消息聚类分析}
  \begin{block}{\textbf{基本思路}}
	  \begin{itemize}
		  \item 使用编辑距离矩阵将类似的消息归于一张连通图中。
		  \item 使用固定值替换感兴趣的消息,如代码、email地址。
		  \item 查找归一化距离小于阈值的消息,并确定聚类边界。
	  \end{itemize}
  \end{block}
\end{frame}

\subsection{算法}
\begin{frame}{算法}
\begin{columns}[T] % align columns
   	\begin{column}{.5\textwidth}
   	\begin{algorithm}[H]  
   		\caption{algorithm caption} %算法的名字
   		\hspace*{0.02in} {\bf Input:} %算法的输入, \hspace*{0.02in}用来控制位置,同时利用 \\ 进行换行
   		input parameters A, B, C\\
   		\hspace*{0.02in} {\bf Output:} %算法的结果输出
   		output result 
   		\begin{algorithmic}[1] %每行显示行号  
   		\State some description % \State 后写一般语句
   		\State ...
   		\State \Return result	
   		\end{algorithmic}  
   \end{algorithm}
   \end{column}

   \hfill

   \begin{column}{.40\textwidth}
 
   \end{column}
   \end{columns}
\end{frame}

\begin{frame}[fragile]{代码}
\fontspec{Consolas}
\begin{lstlisting}
  inline int gcd(int a, int b) { 
    return b==0?a:gcd(b,a%b)
  }

  inline int lcm(int a, int b) {
    return a/gcd(a,b)*b;
  }
\end{lstlisting}
\end{frame}

\section{参考文献}
\begin{frame}{参考文献}
\bibliographystyle{unsrt}
\bibliography{bibs/beamer}
\end{frame}

\begin{frame}{Thank you}
  \begin{center}
    \begin{minipage}{1\textwidth}
	    \setbeamercolor{mybox}{fg=white, bg=black!50!blue}
      \begin{beamercolorbox}[wd=0.70\textwidth, rounded=true, shadow=true]{mybox}
      \LARGE \centering Thank you for listening!  %结束语
      \end{beamercolorbox}
    \end{minipage}
    \end{center}
\end{frame}

\begin{frame}{Q\&A}
\begin{center}
	\begin{minipage}{1\textwidth}
		\setbeamercolor{mybox}{fg=white, bg=black!50!blue}
		\begin{beamercolorbox}[wd=0.70\textwidth, rounded=true, shadow=true]{mybox}
			\LARGE \centering  Questions?  %请求提问
		\end{beamercolorbox}
	\end{minipage}
\end{center}
\end{frame}

\end{document}
